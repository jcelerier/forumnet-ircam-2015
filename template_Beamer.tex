\documentclass[12pt,t]{beamer}
\usetheme{m}

%\usepackage[T1]{fontenc}
%\usepackage[utf8]{inputenc}
\usepackage{microtype}
\usepackage[backend=biber]{biblatex}


\usepackage{nameref}
\makeatletter
\newcommand*{\currentname}{\@currentlabelname}
\makeatother

\beamertemplatenavigationsymbolsempty
\setbeameroption{hide notes}
\setbeamertemplate{note page}[plain]


\title{i-score}
\subtitle{Scoring time and interactivity}
\date{Novembre 2015}
\author{Théo de la Hogue\inst{1} \and Pierre Cochard\inst{2} \and Jean-Michaël Celerier\inst{3}}

\institute{\inst{1} GMEA \and \inst{2} LaBRI - SCRIME \and \inst{3} LaBRI - Blue Yeti}

\begin{document}
   
\maketitle

\begin{frame}
    \frametitle{Table of contents}
    \setbeamertemplate{section in toc}[sections numbered]
    \tableofcontents[hideallsubsections]
\end{frame}

\section{Presentation}
\begin{frame}{i-score}
    \frametitle{i-score}
    \begin{itemize}
        \item Generalist tree-based data sequencer.
        \item Target : authoring of interaction-heavy content.
        \item Applications : interactive shows, music, museography.
        \item Execution semantics based on formal models.
    \end{itemize}
\end{frame}

\subsection{Graphical chart}
\begin{frame}
    \frametitle{\currentname}
        
\end{frame}

\subsection{Features}
\begin{frame}
    \frametitle{\currentname}
    \begin{itemize}
        \item Hierarchy, automations, mappings, custom Javascript execution.
        \item Protocols : OSC, MIDI, Minuit, OSCQuery (in progress).
        \item Multiple plug-in interfaces for extensibility.
        \item Collaborative editing.
        \item Works on OS X, Windows, Linux (desktop and embedded), Android.
        \item Integrated to Max/MSP and command-line player.
        \item Web UI.
    \end{itemize}
\end{frame}


\section{Demo}


\section{Perspectives}
\subsection{Spatial scores}
\begin{frame}
    \frametitle{\currentname}
    Means of authoring spatial-heavy scores.
    
    Examples : audio trajectories, video games, interactive kiosks.
    
    \begin{itemize}
        \item Generic method based on a powerful computer algebra system : GiNaC.
        \item Generalized mapping between any parameters. 
        \item The created structures can influence each other and properties can be 
        extracted (such as collisions, etc.).
    \end{itemize}
\end{frame}

\subsection{Sound}
\begin{frame}
    \frametitle{\currentname}
    \begin{itemize}
        \item Integrating i-score with FaUST or the libaudiostream?
        \item It would allow "Audio" processes that would behave like traditional DAW's tracks.
    \end{itemize}
\end{frame}

\begin{frame}
    Download now !
    \url{https://github.com/OSSIA/i-score/releases}
    Thanks !
\end{frame}


\end{document}
