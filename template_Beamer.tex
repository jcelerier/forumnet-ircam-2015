\documentclass[french,12pt,t]{beamer}
\usetheme{m}

\usepackage{microtype}
\usepackage[backend=biber]{biblatex}


\usepackage{nameref}
\makeatletter
\newcommand*{\currentname}{\@currentlabelname}
\makeatother

\beamertemplatenavigationsymbolsempty
\setbeameroption{hide notes}
\setbeamertemplate{note page}[plain]


\title{i-score}
\subtitle{Écriture du temps et de l'interaction}
\date{Novembre 2015}
\author{Pierre Cochard, Jean-Michaël Celerier}
\institute{LaBRI, SCRIME, Blue Yeti}

\begin{document}
    
    \maketitle

\begin{frame}
    \frametitle{Table des matières}
    \setbeamertemplate{section in toc}[sections numbered]
    \tableofcontents[hideallsubsections]
\end{frame}

\section{Présentation}
\begin{frame}
    \frametitle{\currentname}
    i-score est un séquenceur généraliste, adapté à l'écriture de spectacles interactifs.
    
    
\end{frame}

\subsection{Historique}
\begin{frame}
    \frametitle{\currentname}
\end{frame}


\section{Modèle du temps}
\subsection{Modèle théorique}
\begin{frame}
    \frametitle{\currentname}
    \begin{itemize}
        \item Modèles pour l'édition :
        CSP
        
        \item Modèles pour l'exécution :
        NTCC, Réseaux de Petri, Automates temporisés, langages réactifs
    \end{itemize}
\end{frame}

\subsection{Charte  graphique}
\begin{frame}
    \frametitle{\currentname}
    \begin{itemize}
        \item Syntaxe temporelle
        
        \item Syntaxe logique
        
        \item Syntaxe de données
    \end{itemize}
\end{frame}


\section{Fonctionnalités}
\begin{frame}
    \frametitle{\currentname}
    \begin{itemize}
        \item Hiérarchie, automations, mappings.
        \item Protocoles supportés : OSC, MIDI, Minuit, OSCQuery (en cours).
        \item Interface pour plug-ins.
        \item Répartition à l'édition.
        \item OS X, Windows, Linux...
        \item Intégration Max/MSP, lecteur en ligne de commande.
        \item Contrôle partiel via UIs web.
    \end{itemize}
\end{frame}


\section{Démonstration}


\section{Perspectives}
\subsection{Spatial}
\begin{frame}
    \frametitle{\currentname}
    Écriture adaptée aux scénarios travaillant sur données spatiales.
    
    \begin{itemize}
        \item Méthode générique basée sur CAS GiNaC.
        \item Grande flexibilité pour l'auteur (au prix des performances).
        \item C'est un mapping généralisé qui permet de faire des calculs sur les structures sur lesquelles il opère.
    \end{itemize}
\end{frame}

\subsection{Son}
\begin{frame}
    \frametitle{\currentname}
    \begin{itemize}
        \item Intégration avec FaUST / libaudiostream?
        \item Boites "fichiers audio" avec effets qui s'appliquent dessus.        
    \end{itemize}
\end{frame}

\begin{frame}
    Liens : (github, etc...)
    
    Merci
\end{frame}


\end{document}
